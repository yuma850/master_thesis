\documentclass[10pt]{jarticle}
\usepackage{float}
\usepackage{adrobo_abst}
\usepackage[dvipdfmx]{graphicx}
\usepackage{amssymb,amsmath}
\usepackage{bm}
\usepackage[superscript]{cite}
\usepackage{enumerate}
\usepackage{url}
\usepackage{comment}
%\usepackage[absolute]{textpos}

\setlength\intextsep{5pt}
\setlength\textfloatsep{5pt}

\renewcommand\citeform[1]{(#1)}

\graphicspath{{../fig/}}

\begin{document}
    
    \makeatletter
    \doctype{2023年度卒業論文概要}
    \title{方向可変な車輪を有する8輪運搬ロボットの開発}{}
    \etitle{Development of an 8-wheeled transport robot with variable-direction wheels}{}
    
    \author{20C1013\hspace{.5zw}伊藤優真}
    \eauthor{Yuma ITO}
    
    \makeatother
    
    \abstract{This paper describes the development of a transport robot with a mechanism that can prevent the robot from getting stuck in muddy ground. The method envisioned in this paper requires excavation directly under the structure to create a space up to a certain depth, but the work must be unmanned because of the high-pressure environment in the space, which is harmful to the human body. However, the crawler-type robots currently in use get stuck in muddy areas. We developed the aircraft with the idea that the orientation of the drive unit of the aircraft could be transformed into a circular shape to prevent it from getting stuck.
    }
    
    \keywords{Muddy terrain mobile, Transportation robot, transformer robot}
    
    \maketitle
    
    \supervisor{指導教員:米田完 教授}
    
    \section{緒\hspace{2zw}言}%===========================
    
    
    本研究は, オリエンタル白石株式会社と共同で行い, ニューマチックケーソン工法(以下ケーソン工法)において必要となる運搬ロボットについて研究を行う.

    ケーソン工法とは, 地上に構築した構造物の直下を掘削・排土する事で, 地下に沈設する工法である.
    掘削を行う作業空間は地下水の侵入を防ぐために高圧な環境下であり, 人体に影響が出る為作業の無人化が求められている.
    
    現在はクローラ型のロボットが作業しているが, 作業空間は泥濘地でありスタックを起こす可能性がある.

    クローラ型のロボットがスタックする原因として, 旋回時に履帯の側面で地面を掘ることや, 側面方向の障害物に弱い点が挙げられる.

    泥濘地でのスタックを防止するロボットの開発を目的とする.
    
    \section{研究目標}%===========================
    以下に表記する, 作業空間で想定される環境を走破可能である事を目標とする.

	・土地盤(礫, 砂, シルト, 粘土)(含水率は不定)
	
    ・現場ヤード(アスファルト, コンクリート, 鉄板)

    ・土地盤と現場ヤードの境目などで生じる高低差

    \section{機体構想}
    旋回時に地面を掘る原因として, 各クローラ回転時のベクトルと機体が進むベクトルの角度に大きな差が生じる事で, 横方向に滑る力が働く事が考えられる.
    そして機体が進むベクトルと同方向に推力を加える駆動部や段差踏破機構がない為, スタックしてしまう.
    
    各駆動部の向きを円形に変形させ, 推力のベクトルを円の接線方向に向ける事でスタックを防ぐ事が可能だと考えられる.

    変形する為には駆動部の側面は極力短くして地面を掘る面積を抑える必要がある.
    よって本機体ではクローラではなく車輪を採用し, 駆動部はユニットに分割する方式を採用した.
    変形には機体の両側に多重歯車を使用している.
    機体概要を図 \ref{friction3}に示す.


    \begin{figure}[H]
        \centering
        \includegraphics[width=0.45\textwidth]{image_1_2.png}
        \vspace{-3mm}
        \caption{machine blueprint}
        \label{friction3}
    \end{figure}

    \newpage
    
    \section{機体製作}
    先述の構想を元に製作した機体を図 \ref{prot1}に示す.

    機体のアスペクト比は現在使用されているクローラ型ロボットと同様, 約1.6 : 1 で製作を行った.

    本機体には段差による影響を減らし, 走行時に全ての車輪を接地させる為, 各ユニットごとにサスペンションを搭載した.

    電源を除く機体諸元を表 \ref{specification}に示す.
    

    \begin{figure}[h]
        \centering
        \includegraphics[width=0.45\textwidth]{robot_1_1.png}
        \vspace{-3mm}
        \caption{produced machine}
        \label{prot1}
    \end{figure}

    \begin{table}[h]
        \centering
        \caption{Specification of machine}
        \vspace{1mm}
        \begin{tabular}{cc} \hline
          Length & 870[mm] \\
          Width & 560[mm] \\
          Height & 360[mm] \\ 
          Weight & 57.8[kg] \\ \hline
        \end{tabular}
        \label{specification}
      \end{table}
   

    走行環境は鉄板などの硬質な素材から含水率の高い流体状の地面まで様々である為, 様々な環境に合わせた特殊な形状の車輪を設計した.
    図 \ref{prot2}に示す.

    泥濘地の走行では, 接地面積の少なさを改善するために溝を作り表面積を増やした. さらに走行時に溝で泥を掻く事で, より推力を出す事ができると考える.
    
    硬質な素材上では点接地の方が摩擦を軽減できて走行効率が上がる為, 車輪中央のみ溝を無くしている.


    \begin{figure}[h]
        \centering
        \includegraphics[width=0.45\textwidth]{robot_1_3.png}
        \vspace{-3mm}
        \caption{special shaped wheels}
        \label{prot2}
    \end{figure}



    \section{機体の性能実験}
    今回は現場ヤードでの走行安定性を検証するために, アスファルトと鉄板での走行実験と段差踏破の実験を行った.

    \subsection{走行実験}
    走行実験において走行スピードはモータの性能とプログラムに依存する為, モータの性能から出される理論値との比較を行う. なお今回の実験には機体重量によるモータへの負荷や実験環境による外力は考慮しないものとする.
    
    結果を表 \ref{friction}に示す.
    \vspace{-2mm}

    \begin{table}[h]
        \centering
        \caption{Driving experiment}
        \begin{tabular}{ccc}
        environment           & distance traveled      \\ \hline
        theoretical value     & 26.53[mm/s]            \\
        iron plate            & 24.39[mm/s]            \\
        asphalt               & 17.86[mm/s]
        \end{tabular}
        \label{friction}
    \end{table}
    
    実験の結果, 理論値よりも低い性能ではあったが, 外力の影響も含めると理論値に近い性能を出す事ができた.

    \subsection{段差踏破実験}
    車輪数の多い機体には段差踏破時のタイムロスが考えられるため, 踏破可能な高さと走行時間の実験を行った.
    
    結果を表 \ref{friction2}に示す.
    \vspace{-2mm}

    \begin{table}[h]
        \centering
        \caption{Step-crossing experiment}
        \begin{tabular}{ccc}
        step height    & crossing time   \\ \hline
        6[mm]          & 32[s]           \\
        12[mm]         & 33[s]           \\
        18[mm]         & 32[s]           \\
        24[mm]         & 34[s]
        \end{tabular}
        \label{friction2}
    \end{table}

    実験の結果, 段差の高さによる走行スピードへの影響はほとんどなかった.
    サスペンションにより, タイムロスを抑える事ができたと考えられる.

    \section{結\hspace{2zw}言}
    本研究では泥濘地でのスタックを防止するため, 各車輪の方向を進行方向にそろえる機構をもつ8輪運搬ロボットの提案と製作を行った.
    また現場ヤードを想定した検証を行った.
    
    検証の結果, 平地走行や段差踏破では十分な性能を持つことが検証できた.

    今後は土地盤で含水率を変えつつ走行実験を行い, 泥濘地でのスタック防止機構の有用性の評価を行う.


    \vspace{5truemm}
    {\footnotesize
        \begin{thebibliography}{99}

            \bibitem{magnetic-clawlar2} 
	        工法の概要|オリエンタル白石株式会社
	        
            \url{https://www.orsc.co.jp/tec/newm_v2/ncon02.html}
	        
            (参照日 2024年1月19日)
            
        \end{thebibliography}
    }
    \normalsize
    
\end{document}