\documentclass[11pt,dvipdfmx]{jarticle}
\usepackage{amsmath,amsfonts}
\usepackage{titlesec}
\usepackage{graphicx}
\usepackage{float}
\usepackage{comment}
\usepackage{fancyhdr}
\usepackage{url}
\usepackage[dvipdfmx]{hyperref}
\usepackage{pxjahyper}

\usepackage{setspace}

\usepackage{siunitx}

\usepackage[final]{pdfpages}

\usepackage{bm}
\usepackage[superscript]{cite}
\usepackage{enumerate}
\usepackage{subfigure}
\usepackage{caption}
%\usepackage{fancyhdr} % フッター・ヘッダーを整えるパッケージ
%\usepackage{geometry} % ページ余白を調整するため
%\usepackage[dvipdfmx]{graphicx}
%\usepackage{robomech}

%\lhead{}%ヘッダ左上を空白化
\graphicspath{{../fig/}}%\includegraphicsのファイル名省略用

%高さの設定
\setlength{\textheight}{\paperheight}%ひとまず紙面を本文領域に
\setlength{\topmargin}{-5.4truemm}%上の余白を20mm(=1inch-5.4mm)に
\addtolength{\topmargin}{-\headheight}%
\addtolength{\topmargin}{-\headsep}%ヘッダの分だけ本文領域を移動させる
\addtolength{\textheight}{-40truemm}%下の余白も20mmに
%%幅の設定
\setlength{\textwidth}{\paperwidth}%ひとまず紙面を本文領域に
\setlength{\oddsidemargin}{-0.4truemm}%左の余白を20mm(=1inch-5.4mm)に
\setlength{\evensidemargin}{-0.4truemm}%
\addtolength{\textwidth}{-50truemm}%右の余白も20mmに

%図,表の表示名
\renewcommand{\figurename}{Fig. }
\renewcommand{\tablename}{Table }

%図,表,式などの間隔
\setlength{\abovecaptionskip}{1mm}	%図・表とキャプションの間隔の変更
\setlength{\belowcaptionskip}{1mm}
\setlength{\abovedisplayskip}{3pt}%式の上部のマージン
\setlength{\belowdisplayskip}{3pt}%式の下部のマージン

%図番号を(subsection).(図番号)に変更
\makeatletter
\renewcommand{\thefigure}{\thesection.\arabic{figure}}
\@addtoreset{figure}{section}

%表番号を(subsection).(表番号)に変更
\renewcommand{\thetable}{\thesection.\arabic{table}}
\@addtoreset{table}{section}

%式番号を(subsection).(式番号)に変更
\renewcommand{\theequation}{\thesection.\arabic{equation}}
\@addtoreset{equation}{section}
\makeatother

%注釈
\renewcommand\thefootnote{*\arabic{footnote}}

%目次の表示レベル設定
\setcounter{tocdepth}{3}

\renewcommand{\postpartname}{章} %部を章に変更

\begin{document}

\begin{titlepage}
  \begin{center}

    \vspace{20mm}
    {\Large 千葉工業大学大学院}\\
    \vspace{5mm}
    {\Large 修士学位論文}\\
		\vspace{65mm}
    {\huge IH式加熱装置を搭載した\\\vspace{2mm}鋼橋壁面塗膜除去ロボットの開発}\\
    \vspace{3mm}
    {\Large Development of a robot equipped with an induction heater for removing paint from steel bridge walls}\\
    \vspace{65mm}
		{\Large 2026年3月}\\
    \vspace{10mm}
    {\Large 所属:未来ロボティクス専攻}\\
    \vspace{5mm}
    {\Large 学生番号: 24S1004}\\
    {\Large 氏名:伊藤 優真}\\
    \vspace{5mm}
    {\Large 指導教員:米田 完 教授}
    
  \end{center}

\end{titlepage}

\pagestyle{empty}

%アブストラクト
 %←スペース用全角スペース()
\vspace{20mm}
\renewcommand{\abstractname}{\LARGE 要旨}
\begin{abstract}
  \vspace{3mm}
  本研究では鋼橋のメンテナンスにおける塗膜剥離作業のIH装置による加熱工程を担う磁気クローラ式壁面移動ロボットの開発を行う.
  鋼製の橋は防食性能を維持するために定期的な塗装の塗り替えが必要である. その際, 新たに塗装を行う前に劣化した既存の塗料を除去することが重要である. 現在, 塗膜剥離の方法の一つにIH(電磁誘導加熱)によるものがある. しかし, IHの装置は約12kgと重く, それを保持し操作する作業員の負担の軽減が求められている. そこで, 鋼橋の壁面を磁気クローラで吸着し, IHの装置を自動で移動させることができる壁面移動ロボットの開発した. 磁気クローラは, 履板に設けたピンをレールに沿わせることにより壁面に垂直方向の剛性を持たせる方式を採用している. これは先行研究を参考にした同様の仕組みだが, アキシアル荷重を受けることができる摺動面を設けることにより先行研究では論じられていない旋回動作を考慮した構造としている. 1/2スケールの試作機を製作し, 動作実験・検証を行い, IMUを用いた自律移動の実装も行った. これによって得られた知見をもとにフルスケールサイズの機体の設計製作行った. そして, 実際の橋梁に使用されているものと同様の試験桁において塗膜剥離試験を行い, その有用性を示すとともに課題を明らかにした.
\end{abstract}

\vspace{20mm}
\renewcommand{\abstractname}{\LARGE Abstract}
\begin {abstract}
  In this study, a magnetic crawler-type wall locomotion robot is developed to perform the heating process by IH equipment in the coating removal process in the maintenance of steel bridges.
  Steel bridges require regular repainting to prevent corrosion. 
  When repainting, it is important to remove the deteriorated existing paint before applying new paint. 
  Currently, one method of coating removal is through IH (induction heating).
  However, IH devices are heavy, approximately 12kg, and there is a need to reduce the burden on the workers who operate and hold them.
  Therefore, a wall locomotion robot that can automatically move the IH device by adhering the steel bridge wall surface with a magnetic crawler has been developed.
  The magnetic crawlers are designed to provide vertical rigidity to the wall surface by aligning the pins on the track with the rails. 
  This is a similar mechanism based on previous studies, but by providing a sliding surface that can withstand axial loads, it is a structure that considers turning movement not discussed in previous studies. 
  A half scale prototype was produced and operation experiments and verification were conducted, and autonomous movement using IMU was also implemented.
  Based on the knowledge obtained, the design and production of a full scale robo was carried out.
  Then, The results of the peeling tests on the test girders similar to those used in actual steel bridges show the usefulness of the method and clarify the issues involved.

  \end {abstract}



\newpage
%目次
\pagestyle{fancy}
\renewcommand{\footrulewidth}{0.4pt}%フッターラインの線幅の変更。デフォルトは0.4pt
\pagenumbering{Roman}		%ページ番号をローマ数字に
\tableofcontents

\vspace{5mm}
\leftline{\textgt{付録A: フレーム部品外注図面}}
\leftline{\textgt{付録B: 履板外注図面}}
\leftline{\textgt{付録C: 駆動系部品外注図面}}
\leftline{\textgt{付録D: 駆動系部品小野電機修正案}}
\leftline{\textgt{付録E: 外注部品見積書}}
\leftline{\textgt{付録F: 吸着ロボットによる高周波誘導加熱(IH)式塗膜剥離の試行(案)}}

\newpage
%図目次の表示
\listoffigures
%表目次の表示dd
\listoftables

\clearpage
\setcounter{page}{1}
\pagenumbering{arabic}	%ページ番号をアラビア数字に

%第1章
%第1章
\newpage
\part{序論}
\label{zyoron}

\section {日本の鋼橋と塗膜剥離工法}
	%\subsection{研究背景}

	日本には多くの鋼橋が存在する.
	道路橋は15[m]以上のものだけでも約6万橋, 鉄道橋は4万橋存在する(2011年時点)\cite{koukyou}.
	鋼橋は腐食や経年劣化により損傷を起こすため,表面に塗装を施すことで防いでいる.
	防食性能を維持するには定期的な塗装の塗り替えが必要だが,塗装前に劣化した塗膜を剥離させなければならない.
	塗膜を剥離させる方法として,高圧の空気で研削材を噴射して塗装を剥がす塗膜剥離ブラストや塗膜剥離剤を用いた工法が挙げられる.
	しかし,塗膜剥離ブラストは騒音や粉塵の問題があり\cite{burasuto},塗膜剥離剤には臭気や有害化学物質(ベンジルアルコールやジクロロメタンなど)が発生する物もあるなど\cite{hakurizai},いずれも作業員の人体に影響を与えることが多い.
	近年ではそのような問題を解決するため,IH式塗膜剥離工法が普及し始めている.


	\subsection{IH式塗膜剥離工法}
		IH式塗膜剥離工法とは,鋼板を加熱することで塗装と鋼板面の接着が緩まり,塗膜を容易に剥離・除去することが可能になる工法である\cite{IHkouhou}
		鋼板の加熱にはIH(電磁誘導加熱)装置システムを使用しており,電磁誘導により加熱を行う.
		IH装置システムとは,ノルウェーのRPRテクノロジー社が開発した高周波誘導加熱塗膜除去装置「RPR1650システム」\cite{RPR1650}と日本橋梁株式会社が製作した「冷却水供給ユニット」を組み合わせたシステムである.\cite{IHsouchi}
		システムの構成図を図 \ref{IHsouchikousei}に示す.

	\begin{figure}[H]
		\centering
		\includegraphics[width=0.8\textwidth]{ih-system.png}
		\caption{System diagram of the induction heating paint removal\cite{IHsouchi}}
		\label{IHsouchikousei}
	\end{figure}

		工法の手順としては,まず図\ref{IHsouchikousei}におけるインダクションヘッドとハンドヘルドトランスフォーマー組み合わせたもの(以下加熱装置)を作業者が移動させることで,塗装の加熱を行う.
		加熱することで塗装が界面破壊をおこし,浮き上がった塗膜にスクレーパを差し込むことで,塗膜の除去が可能となる.
		IH式塗膜剥離工法の図 \ref{IHkouhou}に示す.
		
	
	\begin{figure}[H]
		\centering
		\includegraphics[width=0.8\textwidth]{IHkouhougaiyou.png}
		\caption{Overview of the induction heating paint removal}
		\label{IHkouhou}
	\end{figure}

	\subsection{工法の課題点}
		IH式塗膜剥離工法は従来の塗膜剥離工法に比べ,騒音や粉塵,臭気などの発生が少なく,作業環境の改善に寄与する.
		しかし,加熱装置はコンデンサーボックスを除いた状態で約15kgほどあり,作業者はそれを手動で移動させる必要がある.
		鋼橋は,形式にもよるが作業範囲が広いため基本長時間の作業であり,さらに高いところを加熱する際は装置を頭上に持ち上げた状態で維持しなければならないため,作業員には非常に負担のかかる作業となっている.
		また,作業者の技量によって加熱速度や加熱ムラが発生したり,スクレーパの使い方による塗膜除去の品質にばらつきが出ることも課題である.
		重量物を用いた作業や長時間一定動作を繰り返す作業などは,機械が得意とする内容であるため,工法の機械化が望まれている.
		作業員が行っている作業の様子を以下の図\ref{zinnrikisagyou}に示す.

	\begin{figure}[H]
		\centering
		\includegraphics[width=0.8\textwidth]{ih-work.png}
  	\caption{Work of the induction heating paint removal\cite{ih-catalog}}
		\label{zinnrikisagyou}
	\end{figure}

	\subsection{研究目的}
		IH式塗膜剥離工法では,機械による加熱装置の壁面移動と加熱後の塗装の除去が求められている.
		そこで本研究では,加熱装置を搭載した状態で鋼橋の壁面を移動可能であり,かつ塗装の除去を行うことが可能なロボットの開発による,作業員の身体的負担の軽減を目的とする.
		本研究は日本橋梁株式会社(以下日本橋梁)との共同研究である.


\newpage

\section{現場環境とこれまでの取り組み}

	\subsection{鋼橋の種類と現場環境}
		鋼橋は形式によって塗装対象となる構造部材や作業環境が大きく異なる。
		例えば、トラス橋や斜張橋に代表される高さ方向に規模のある鋼橋では、鋼製橋脚や主塔、トラス部材など、鉛直方向に広範囲な塗装面を有する。
		一方、プレートガーダ橋(I桁橋・箱桁橋)に代表される横方向に長い鋼橋では、主に鋼桁下面を中心とした水平方向に広がる塗装面が対象となる。
		このように、鋼橋の形式によって塗装面の分布や作業姿勢が異なるため、塗膜除去作業の効率化には橋梁形式に応じた機構設計が求められる。
		それぞれの橋の構造を以下の図\ref{bridge_types}に示す。

	\begin{figure}[H]
		\centering
		\includegraphics[width=0.8\textwidth]{bridge_types.png}
  	\caption{Work of the induction heating paint removal\cite{koukyousyurui}}
		\label{bridge_types}
	\end{figure}


\newpage
	\subsection{先行研究}
		これまで行われてきた研究では,基本的にロボットが装置を搭載した状態で壁面に吸着・移動する方式で,壁面の加熱のみを行うロボットが開発されてきた.
		塗膜除去まで行い,工法の一連の作業を行う研究は行われていない.
		以下に代表的な先行研究を示す.


		\subsubsection{Vertical \& Horizontal Robotics\cite{rpr-robot}}
			本研究でも使用するIH塗膜剥離装置の開発元であるRPR Technologiesが提案するロボットである.
			貯蔵タンクに磁石で吸着し, IHヘッドを自動で移動させ加熱することができる.
			機体が大型であるため, 本研究が目的とする橋梁の塗膜剥離作業に運用することは難しい.

		\begin{figure}[H]
			\centering
			\includegraphics[width=0.5\textwidth]{rpr-robot.jpg}
  		\caption{Vertical \& Horizontal Robotics\cite{rpr-robot}}
			\label{fig:ih4}
		\end{figure}

		\subsection{メカナムホイール式壁面走行ロボット\cite{mekanamu-kyuuchyaku}}
			米田研究室 綱川が開発したメカナムホイールを用いた壁面移動ロボットである.
			壁面吸着には磁石を使用しており,メカナムホイールを用いる事で壁面上での全方向移動が可能となっている.
			
			壁面に吸着し移動するため作業範囲に制限がなく,橋脚のような高所での作業に適している.
			しかし壁面に対して接地線が4つのみであるため,障害物に対して弱く落下の危険性がある.

		\begin{figure}[H]
			\centering
			\includegraphics[width=0.5\textwidth]{hitachi.jpg}
  		\caption{Wall Surface Robot with Magnetic Crawlers\cite{magnetic-clawlar3}}
			\label{fig:ih5}
		\end{figure}

\newpage
		\subsubsection{磁気クローラ式吸着移動ロボット\cite{magnetic-clawlar}~\cite{magnetic-clawlar3}}
			米田研究室 青木が開発した磁気クローラを用いた壁面移動ロボットである.
			壁面吸着には磁石を使用しており,各履板全てに磁石を配置することで高い吸着力を実現している.
			
			こちらの機体も壁面に吸着し移動するため作業範囲に制限がなく,橋脚のような高所での作業に適している.
			しかし,横移動には必ず旋回動作が必要となるが,吸着力の高さとクローラの性質上浮き上がった塗膜に影響が出る可能性がある.
			
		\begin{figure}[H]
			\centering
			\includegraphics[width=0.5\textwidth]{hitachi.jpg}
  		\caption{Wall Surface Robot with Magnetic Crawlers\cite{magnetic-clawlar3}}
			\label{fig:ih5}
		\end{figure}
		
	
	\subsection{研究目標}
	\label{goal}
		研究目的である作業員の身体的負担の軽減を達成するには,塗装の加熱から除去までの一連の動作をロボットが行う必要がある.
		また先行研究での課題点も踏まえ以下の目標を設定する.

	\begin{description}
		\item[塗装の加熱から除去までの一連の動作が可能]\mbox{}\\
		\vspace{-3mm}
		\item[高さのある橋脚上での作業が可能]\mbox{}\\
		\vspace{-3mm}
		\item[横方向に長い桁上での効率的な作業が可能]\mbox{}\\
		\vspace{-3mm}
		\item[塗装の状態に影響されない,かつ塗装の状態に影響をあたえない,かつ落下の危険性が低い移動が可能]\mbox{}\\
	\end{description}

	\vspace{-5mm}
		また共同研究先の日本橋梁から以下の要望もあるため,これらも達成すべき研究目標とする.

	\begin{description}
		\item[安全性と効率を両立する]\mbox{}\\
		\vspace{-3mm}
		\item[端部検出による停止]\mbox{}\\
		\vspace{-3mm}
		\item[半自動で塗膜剥離が可能]\mbox{}\\
		\vspace{-3mm}
		\item[2人で運搬可能]\mbox{}\\
		\vspace{-3mm}
		\item[約0.05 m/sで加熱しながら移動]\mbox{}\\
	\end{description}

\section{本論文の構成}
	本論文は以下の構成で記述する.
	本研究の目標である安全性と効率の両立を,高さのある橋脚と横方向に長い桁上の両方の環境で達成することは困難であると考えた.
	そこで本研究では,それぞれの環境に合わせた2種類のロボットを開発し,論じることとする.

	第\ref{zyoron}章では,研究背景と目的を示し,それに対する従来の研究を踏まえた上で,本研究のなすべき目標を明確に示した.

	第\ref{waiyaturisagesiki}章では,高さのある橋脚での作業を想定したワイヤ吊り下げ式壁面塗膜除去ロボットについて述べる.

	第\ref{sinsyukugata}章では,横方向に長い桁上での作業を想定した伸縮型壁面塗膜除去ロボットについて述べる.

	第\ref{zissyouzikkenn}章では,現場を想定した環境での実証実験とその結果について述べる.

	第\ref{conclusion}章では,本研究のまとめと今後の課題について述べる.

%第2章
%第2章
\newpage
\part{ワイヤ吊り下げ式壁面塗膜除去ロボット}
\label{waiyaturisagesik}

\section{機体構想}
	本章で説明するロボットは高さのある橋脚での作業を想定している.
	高さ方向に広く作業範囲がある場合,先行研究と同様に壁面に吸着した状態で移動する方式が優位だと考えられる.
  
	鋼橋での壁面移動の方式は大きく分けて磁気吸着移動・真空吸着移動・ワイヤ移動の3パターンがある.

	磁気吸着移動と真空吸着移動は壁面の状態に影響を受けやすく,壁面の凹凸によっては落下の危険性もある.
	またIH塗膜剥離装置は200度近い温度で加熱を行うため,熱に弱い磁石では吸着力が低下していき,ブロアや真空ポンプを用いた真空吸着では,電源喪失した際に落下してしまう.

	それに比べワイヤを用いた壁面移動では,ワイヤを介してレールなどに常に接続された状態であるため落下の可能性がかなり低い.
	また吊り下げ式であれば壁面との接触面を抑えることも可能であるため,壁面の状態による影響を受けにくく,さらに高負荷をかけての壁面移動も可能である.
	よって本章のロボットは,吊り下げ式かつワイヤのみで壁面を全方位移動可能なロボットである.

	\subsection{動作原理の検討}
		ワイヤのみで壁面上の全方位を移動可能にするため,以下の図\ref{label2_2}のような組み方を考案した.

		図の黒線がワイヤを示している.
		上昇下降の動作は全てのプーリを同方向に回転させてワイヤを巻き取ることで行う.
		左右方向に動作する場合,上下プーリは固定した状態で機体の姿勢を維持する.
		そして左右プーリを逆方向に回転させることで巻き取り動作と巻き出し動作を別々に行い,左右任意の方向に移動を行う.

		これにより壁面上の全方位を移動可能となると考えられる.

		さらにこの張り方の最大の利点が,張られている全てのワイヤの内いずれか1本が破断したとしても落下の可能性が限りなく低いことである.
		上下プーリでロボットを水平に維持し左右プーリは左右移動に使われるが,いずれもロボットを吊り下げて支えておりレールランナー同士の間隔もワイヤで制限をかけているため,破断にも対応可能だと考えられる.

	\begin{figure}[H]
		\centering
		\includegraphics[width=0.8\textwidth]{dousaimage.png}
  	\caption{moving image}
		\label{label2_2}
	\end{figure}

\newpage

	\subsection{機体構成}
		
	ロボットの構想を図 \ref{label2_1}に示す.
	機体の構成は,移動用のワイヤ巻き取りプーリが4つ,それらを駆動させるためのアクチュエータが4つ,プーリに接続されたワイヤーが機体上部のレールランナーに接続されて機体を支え,レールランナーがレールに沿いながら横方向へ移動を行う.
	壁面との距離調整のために機体の4点にねじ式のボールキャスターを搭載している.

  \vspace{2mm}
	\begin{figure}[H]
		\centering
		\includegraphics[width=0.8\textwidth]{robotkousou.png}
  	\caption{machine image}
		\label{label2_1}
	\end{figure}

  	\subsubsection{加熱装置と搭載方法}
			ロボットに搭載する加熱装置は第\ref{zyoron}章の\ref{IHsouchikouse}で示したIH装置システムの先端部分にあたるインダクションヘッドとハンドヘルドトランスフォーマーを組みあわせたものである.
			加熱装置を図\ref{kanetusouti}に示す.

		\vspace{2mm}
		\begin{figure}[IH system]
			\centering
			\includegraphics[width=0.8\textwidth]{kanetusouti.jpg}
  		\caption{IH}
			\label{label2_1}
		\end{figure}

			壁面の加熱は図\ref{kanetusouti}で下向きとなっているインダクションヘッドの先端で行うため,加熱装置は壁面に対して水平に搭載する必要がある.
			よって重心位置は壁面に対して高さを抑えた壁よりの位置にすることができる.

	\newpage

  	\subsubsection{ワイヤ巻き取り用プーリ設計}
			ワイヤは強度に優れており高荷重を支えるのに適しているが,使用上で最も問題となるのが巻き取りの仕方である.

			何もない曲面で負荷をかけながらワイヤを巻き取る場合,途中でワイヤが乱巻きを起こす可能性が高く,各プーリを同じだけ回転させても巻き取り量が変わるため移動に影響を与える.
			さらに乱巻きが起こると,巻き出しの際に出すことができず移動できなくなる可能性がある.

			そこで図 \ref{label5}のような溝付きプーリの設計を行った.
			諸元を表 \ref{specification3}に示す.

			プーリの曲面に溝を付けることで,機体重量により張力があるためワイヤが溝に沿ってまかれ,乱巻きを防止することが可能である.
			1層目を溝に沿って綺麗にまくことができれば,2層目以降は隣り合うワイヤが溝を作るため,常に溝に沿って巻きつけることが可能である.

			今回開発したロボットの作業範囲は基本レールやフレームに依存するが,ワイヤの巻き取り量によっても変わってくるため,プーリ径も重要な値となる.
			下記の式により,プーリの外径とそれによる最大可動高さを導出可能である.

			(1)は使用するプーリの値から周長分のワイヤ長さを導出している.
			(2)はプーリに巻くワイヤの段数によって必要なプーリの壁の外径を導出している.
			(3)は(2)より求められたプーリの壁の外径を元に,ロボットの最大可動高さを導出している.

			今回のロボットに使用しているプーリの壁の外径 $D$ は78[mm]であるため,下記の式より理論上の最大可動高さは2863[mm]である.
			これはプーリにワイヤを10段重ねた際の値である.

			これにより,理論上は研究目標である約2[m]の高さでの作業が可能である.




			%また,ワイヤは壁面と平行な状態で巻き取る必要があるため,プーリから直接レールに繋げることはできない.
		
			%よって,図 \ref{label5_2}のように補助プーリを,プーリより壁際でかつレールより垂直で真下の位置に配置する.
			%これにより,巻き取る際にも常にワイヤは壁面と水平を維持することが可能である.
			%また補助プーリはリニアブッシュを用いて横方向にも移動可能であるため,巻き取る際に溝に沿ってきれいに巻くことを補助する役割も担っている.


    \begin{figure}[h]
			\centering
			% 画像
			\begin{minipage}{0.19\linewidth}
					\centering
					\includegraphics[width=\linewidth]{puurisekkei.jpg}
					\caption{Pulley design}
					\label{label5}
			\end{minipage}
			\hspace{1mm} % 間隔を設定
			% 表
			\begin{minipage}{0.3\linewidth}
					%\centering
					\captionof{table}{Pulley specifications}
					\begin{tabular}{cc} \hline
						$d$ :Pitch circle diameter & 58 [mm] \\
						$D$ :Pulley outer diameter & 78 [mm] \\
						$n$ :Number of grooves  & 12 [本]\\
						$a$ :Wire diameter    & 2 [mm] \\ \hline
					\end{tabular}
					\label{specification3}
			\end{minipage}
			\hspace{-5mm} % 間隔を設定
			% 数式
			\begin{minipage}{0.45\linewidth}
				%\centering
				\begin{itemize}
					\item[]\mbox{}
					\vspace{0mm}
					\begin{equation}
						L_{x} = (d + xa)π
						%\label{eqn: eq1}
					\end{equation}
					\begin{equation}
						D \geq d + x a
						%\label{eqn: eq1}
					\end{equation}
					\begin{equation}
						H_{MAX} = D π n
						%\label{eqn: eq1}
					\end{equation}
	
					%\centering
					\hspace{12mm} $L_{x}$ \hspace{2mm} : Pulley circumference
					
					\hspace{12.2mm} $x$ \hspace{3.6mm} :  \hspace{0mm} Number of stages
					
					\hspace{9mm} $H_{MAX}$ \hspace{-1.8mm} :  \hspace{0mm} Maximum movable height \hspace{1mm}
					
				\end{itemize}
	
			\end{minipage}
		\end{figure}
	
  
  	\subsubsection{レール固定方法の検討}


		\subsubsection{塗膜除去機構の検討}


	\subsection{機構の有意性}


\newpage

\section{製作機体}

	\subsection{機体概要}
		実際に使用されるIH塗膜剥離装置を搭載した機体の製作を行った.
		吊り下げ機体を図 \ref{label3_1}に, 全体図を図 \ref{label3_2}に,諸元を表 \ref{specification1}に示す.

		図 \ref{label3_1}に示すように,吊り下げ機体の外殻はアルミフレームで構成しているため,外部からの衝撃などからIH塗膜剥離装置やアクチュエータ類を保護することができる.

		また,アクチュエータの配置も図 \ref{label3_1}の通りであり,全て吊り下げている機体内にまとめることが可能である.
		IH塗膜剥離装置はヘッドの部分しか加熱できないため,このようなアクチュエータの配置にすることで機体の全幅を抑え,より効率良く加熱することが可能となる.

		作業範囲については,図 \ref{label3_2}に示すようにレールやフレームに依存するが,取り外しが可能なため作業現場に合わせて範囲を拡大または縮小可能である.

  \begin{figure}[h]
	\centering
	% 画像1
	\hspace{-20mm} 
	\begin{minipage}{0.4\linewidth}
			\centering
			\includegraphics[width=\linewidth]{robotturisagebu.png}
			\caption{Produced machine}
			\label{label3_1}
	\end{minipage}
	\hspace{10mm} % 画像1と画像2の間のスペース
	% 画像2
	\begin{minipage}{0.4\linewidth}
			\centering
			\includegraphics[width=\linewidth]{robotzenntai1.png}
			\caption{Overall diagram of the robot}
			\label{label3_2}
	\end{minipage}
	\hspace{10mm} % 画像2と表の間のスペース
	% 表
	\begin{minipage}{0.5\linewidth}
			\centering
			\captionof{table}{Specification of machine}
			\scalebox{1.5}{
			\begin{tabular}{cc} \hline
							Length & 693 [mm] \\
							Width & 450 [mm] \\
							Height & 215 [mm] \\
							Weight & 40.3 [kg] \\ \hline
			\end{tabular}
			}
			\label{specification1}
	\end{minipage}
\end{figure}

	\newpage
  \subsection{ロボットと壁面間の距離調整}
    IH塗膜剥離装置はIHヘッドが壁面に非接触な状態で加熱を行うが,十分な加熱を行うために壁面との適切な距離を常に一定に保つ必要がある.
		また壁面を走行する際,重力と機体重量により壁面から剥がれる方向にモーメント力が発生する.
		さらに,加熱後の塗膜は浮き上がるなど壁面の状態は一定ではないため,壁面との接触面積は可能な限り小さくする必要がある.

		これらの条件をネジ式ボールキャスタとネオジム磁石を使用することで解決した.

		壁面間距離を常に一定に保ち,かつ壁面との接触面積を小さくするため,ネジ式ボールキャスタをロボットの端4点に搭載した.
		ネジ式であるため4点それぞれの高さをネジのピッチで調整し,機体と壁面の距離を適切に保つことが可能である.

		そして壁面から剥がれる方向のモーメント力を打ち消すため,機体上部の2箇所に強力なネオジム磁石を壁面と非接触な状態で搭載した.
		これにより磁力でモーメント力を打ち消すことが可能であり,かつ壁面と非接触であるため,本論文の機体構想にも記載した熱による磁力の低下も抑えることが可能である.

		またワイヤを使用しているため,磁力が低下し吸着が不可となった場合でも落下の可能性は低いと考えられる.

		磁石を付けていない状態の機体側面を図 \ref{label4_1}に, 磁石を付けた状態の機体側面をを図 \ref{label4_2}に,吸着時の壁面間距離を表 \ref{specification2}に示す.

    \vspace{0mm}
		\begin{figure}[h]
			\centering
			% 画像1
			\begin{minipage}{0.26\linewidth}
					\centering
					\includegraphics[width=\linewidth]{zisyakunasi.JPG}
					\caption{Side view of robot without magnet}
					\label{label4_1}
			\end{minipage}
			\hspace{5mm} % 画像1と画像2の間のスペース
			% 画像2
			\begin{minipage}{0.27\linewidth}
					\centering
					\includegraphics[width=\linewidth]{zisyakuari.png}
					\caption{Side of robot with magnet attached}
					\label{label4_2}
			\end{minipage}
			\hspace{5mm} % 画像2と表の間のスペース
			% 表
			\begin{minipage}{0.3\linewidth}
					\centering
					\captionof{table}{Wall distance}
					\scalebox{1.3}{
					\begin{tabular}{c|c}
							      & Wall distance \\ \hline
										IH head & 3 [mm] \\
										Magnet & 1 [mm] \\
					\end{tabular}
					}
					\label{specification2}
			\end{minipage}
		\end{figure}


	\subsection{レール固定部の構造}


	\subsection{塗膜除去機構}


\newpage

\section{機体の性能評価実験}
  研究目標を達成しているかの評価実験を行う.
	
	製作機体より,加熱装置約12[kg]を搭載可能,加熱装置装置と壁面間の距離を一定に保つことが可能の2点は達成しているため,加熱は問題なく行うことができると考えられる.
	よって,一定の速度で移動し作業することが可能であるか,また想定外の外力にも対応可能で安全性が確保されているかの2点に着目した評価実験を行う.


  \subsection{加熱装置の移動実験}
    加熱装置を搭載した状態での動作実験を行う.
		
		横方向1[m],縦方向2[m]の作業範囲を動作することが可能であるか,また動作にかかる時間と速度の計測を行った.
		横方向の移動でかかる時間は1[m]での計測だが,実験結果の表ではデータを合わせるため2倍の値としている.

	\vspace{1mm}


    \subsubsection{実験結果}
      IH塗膜剥離装置を搭載した状態での動作は可能であった.
		  得られたデータを表 \ref{data1}に示す.
			
		  表 \ref{data1}より,各方向別での移動速度に多少の差が見られた.とくに上昇時に大きな速度低下が見られた.

		  人が作業を行う際には,約50[mm/s]の速度で作業を行うことができるため,ロボットの速度を調整する必要がある.

		
      \vspace{-2mm}
	  \begin{table}[h]
			\centering
			\caption{Operation experiment results}
			\scalebox{1.0}{
		  \begin{tabular}{c|cccc}
          & Rise     & Descent   & Lateral movement on top         & Lateral movement at the bottom   \\ \hline
					Speed       & 30.58 [mm/s]        & 40.00 [mm/s]        & 34.61 [mm/s]                    & 37.50 [mm/s]        \\
					Time        & 65.38 [t]           & 50.00 [t]           & 57.77 [t]                       & 53.33 [t]        \\
		  \end{tabular}
		  }
		  \label{data1}
	  \end{table}

	  \vspace{-2mm}
  


    \subsubsection{考察}
      実験結果より,移動速度に多少の差があり,とくに上昇時に大きな速度低下が見られた.
			これは,上昇時には巻き取りの影響により機体が傾くため,その補正を行いながら移動する必要があるからだと考えられる.

			また全体的に移動速度が遅い原因は,このロボットはコントローラで人の手による動作だからだと考えられる.

	\subsection{レール固定部耐久実験}
	
  \subsection{電流値計測試験}
    ロボットの十分な安全性を評価するため,電流値計測試験を行う.

		ロボットの安全性評価にはさまざまな内容があるが,本研究のロボットは4本のワイヤで吊り下げているため,それぞれのワイヤにかかっている負荷を計測する.
		ワイヤの張力を計測する必要があるが,ワイヤの張力は直接モータの負荷となるため,モータに流れる電流値から計測可能である.

		実験を行うにあたり,各モータを番号付けする必要がある.
		図 \ref{label2_1}で表すと,左右プーリの左側が1,右側が2,上下プーリの左側が3,右側が4として番号付けしている.
		また動作は停止状態から上昇,下降,上昇,左側移動,右側移動の順で行っている.

	\vspace{1mm}


    \subsubsection{実験結果}
      実験結果としては,主に左右プーリのモータと上下プーリのモータで電流値の変化が分かれた.
		  また各動作ごとでも電流値に変化が見られた.
		  得られたデータを図 \ref{label6}に示す.

		  上昇時には全てのモータに大きな電流が流れたが,特に上下プーリのモータに負荷が見られた.
		  下降時は基本どのモータも流れる電流はとても少なかった.
		  左右移動時はそれぞれのモータの負荷に変化があった.
		  上下プーリのモータは進行方向側のモータには大きな電流が流れ,反対側のモータに流れる電流は少なかった.
		  これは進行方向によって対照的な変化が見られた.
		  また左右プーリのモータは進行方向に関わらず,ほとんど電流は流れなかった.

      \vspace{-1mm}
		\begin{figure}[h]
			\centering
			\includegraphics[width=1.0\textwidth]{zikkenn2.png}
			\vspace{-2mm}
			\caption{Current value measurement results}
			\label{label6}
		\end{figure}

	  \vspace{-3mm}


    \subsubsection{考察}
      実験結果より,上昇時に全てのモータに大きな電流が流れた理由はロボットの重量によるものであると考えられる.
		  上下プーリ用モータと左右プーリ用モータで電流値が変わってきた理由は,左右プーリの方がワイヤを多く巻き出しているからだと考えられる.
		  これにより全てのプーリを同じ速度で巻き取ると,プーリの周長が変わっているため巻き取り量が少なくなってしまい,負荷が減るのではないかと考えられる.

		  下降時は機体重量が移動のサポートをするため,流れる電流が少なくなるのだと考えられる.

		  左右移動時に全てのモータの電流値が変化する理由は,本研究のロボットは上下プーリ用モータで基本姿勢を維持するため,片側が巻き取られると姿勢維持しているプーリに負荷がかかるからだと考えられる.
		  また左右移動用のプーリに負荷がかからなかった理由は,上記と同様で基本姿勢維持は上下プーリで行っているからだと考えられる.

		  全体的に見て電流値に違いはあるが,それぞれ対照に負荷がかかっており偏りがないため,安全であると考えられる.
	

  \subsection{ワイヤ破断試験}
    壁面移動ロボットにおいて安全性を証明するには,想定外の外力にも対応可能である必要がある.
		機体構想でも記載した通り,本研究のロボットはいずれかのワイヤが破断しても落下することはないと考えられる.
		よって実際にワイヤを取り外す実験を行う.

		実際に破断させることは難しいため,この実験ではワイヤの拘束を解除することで代用している.

		全てのワイヤを1本ずつ拘束を解除し,ロボットの位置の変化を計測する.
		計測は拘束解除後10[s]の状態で行う
		
		また本研究のロボットは上下プーリ用ワイヤか左右プーリ用ワイヤのどちらかが2本切れた場合も落下しないと予想できる.
		こちらの実験も同様に行う.


    \subsubsection{実験結果}
      実験の結果, 構想段階での予想通り,いずれのワイヤが破断しても落下することはなかった.
		  また上下プーリ用ワイヤか左右プーリ用ワイヤのどちらかが2本破断した場合の実験でも同様に落下することはなかった.

		  詳しい状態としては,上下プーリの右側のワイヤを破断させた場合,機体が外した側に傾いたがその場から位置がずれることは無かった.
		  上下プーリの左側のワイヤを破断させた場合も同様である.

		  左右プーリの右側のワイヤを破断させた場合,外した側と反対方向に機体の位置が横ずれを起こしたが,機体が傾くなどは無かった.
		  こちらも,反対側でも同様の状態となった.

		  また,上下プーリの両側のワイヤを破断させた場合と,左右プーリの両側のワイヤを破断させた場合は,どちらも機体の姿勢や位置に変化はなかった.

		  上下プーリの右側のワイヤを破断させた状態を図 \ref{label7_1}に,左右プーリの右側のワイヤを破断させた状態を図 \ref{label7_2}に,
		  上下プーリの両側のワイヤを破断させた状態を図 \ref{label7_3}に,左右プーリの両側のワイヤを破断させた状態を図 \ref{label7_4}に示す.

      \begin{figure}[H]
			\centering
			% 画像1
			\begin{minipage}{0.2\linewidth}
					\centering
					\includegraphics[width=1.0\linewidth]{zikkenn3_1.png}
					\caption{Wire breakage condition 1}
					\label{label7_1}
			\end{minipage}
			\hspace{5mm} % 画像1と画像2の間のスペース
			% 画像2
			\begin{minipage}{0.2\linewidth}
					\centering
					\includegraphics[width=0.92\linewidth]{zikkenn3_2.png}
					\caption{Wire breakage condition 2}
					\label{label7_2}
			\end{minipage}
			\hspace{5mm} % 画像1と画像2の間のスペース
			\begin{minipage}{0.2\linewidth}
				\centering
				\includegraphics[width=1.0\linewidth]{zikkenn3_3.png}
				\caption{Wire breakage condition 3}
				\label{label7_3}
			\end{minipage}
			\hspace{5mm} % 画像1と画像2の間のスペース
			% 画像2
			\begin{minipage}{0.2\linewidth}
					\centering
					\includegraphics[width=1.0\linewidth]{zikkenn3_4.png}
					\caption{Wire breakage condition 4}
					\label{label7_4}
			\end{minipage}

		\end{figure}


    \subsubsection{考察}
      実験結果より,上下プーリの片側のワイヤを破断させて機体が把持した側に傾いた理由としては,上下プーリが主体でロボットを支えており姿勢維持を担っているため,片側が破断するとバランスを崩すからだと考えられる.

		  左右プーリの片側のワイヤを破断させて機体が反対側に位置ずれした理由としては,上下プーリで姿勢維持をしているが,左右に引っ張る力のバランスが崩れたからだと考えられる.

		  上下と左右のプーリの両側を破断させた場合に機体が位置ずれも傾きも起こらなかった理由としては,破断後も左右対称であるため均等に機体の荷重を支えることができているからだと考えられる.

		  この結果により,安全性が証明されたと言える.
%第3章
%第3章
\newpage
\part{伸縮型壁面塗膜除去ロボット}
\label{sinsyukugata}

\section{機体構想}

    \subsection{機体構成}

        \subsubsection{加熱装置と搭載方法}

            \textgt{付録A~C}に機体部品の外注図面を掲載した.

        \subsubsection{伸縮方式の検討}

        \subsubsection{塗装除去機構の検討}

        \subsubsection{平面移動方式の検討}

    \subsection{構想の有意性}

\section{製作機体}

    \subsection{機体概要}
    
    \subsection{伸縮機構}

    \subsection{壁面距離調整機構}

    \subsection{全方向移動機構}

\section{機体の性能評価}

    \subsection{加熱装置の昇降実験}

        \subsubsection{実験内容}

        \subsubsection{実験結果・考察}

    \subsection{塗膜除去機構の貫入力実験}

        \subsubsection{実験内容}

        \subsubsection{実験結果・考察}

    \subsection{平面移動安定性の評価実験}

        \subsubsection{実験内容}

        \subsubsection{実験結果・考察}
 
%第4章
%第4章
\newpage
\part{現場を想定した動作試験}
\label{zissyouzikkenn}

\section{まとめ}
\section{課題}
\subsection{履板や磁石の熱による影響}
\subsection{補強材や桁間における取り付け・取り外し}
\section{今後の展望}
\section{発展的な展望}
%第5章
%第5章
\newpage
\part{結論}
\label{conclusion}


\section{まとめ}
\section{課題}
\section{今後の展望}
\section{発展的な展望}

%参考文献
\newpage
\lhead {}   %:天左側ヘッダの定義
\addcontentsline{toc}{section}{参考文献}
\begin{thebibliography}{99}
	\bibitem{koukyou} 山田健太郎, 舘石和雄. 鋼橋の維持管理, コロナ社, 2015, 1p.

	\bibitem{paint} 鋼道路橋塗装・防食便覧資料集, 社団法人日本道路協会, 2010

	\bibitem{youtuu} 職場における腰痛予防対策指針, 厚生労働省 
	\newline
	\url{https://www.mhlw.go.jp/stf/houdou/2r98520000034et4-att/2r98520000034pjn_1.pdf}
	(閲覧日2023-1-10)

	\bibitem{ih-catalog} 
	IH(電磁誘導加熱)装置システムによるIH式塗膜剥離, 日本橋梁株式会社, 
	\url{https://www.nihon-kyoryo.co.jp/assets/pdf/03_JB-IHremover.pdf}
	(閲覧日2023-1-10)
	
	\bibitem{ih-group} 
	IH塗膜除去工法研究会 http://ih-tmk.com/
	(閲覧日2023-1-10)
	
	\bibitem{rpr-robot} RPR Induction Disbonding System, RPR Technologies 
	\url{http://www.rprtech.com/wp-content/uploads/2015/02/RPR-Green-Tank-Introduction.pdf}
	(閲覧日2023-1-10)
	
	\bibitem{magnetic-clawlar} 
	内藤紳司. 磁気クローラ式吸着移動ロボット. 日本ロボット学会誌. 1992, vol.10,pp.606-608.
	\url{https://doi.org/10.7210/jrsj.10.606}
	(閲覧日2023-1-10)
	
	\bibitem{magnetic-clawlar2} 
	内藤紳司. 佐藤主税, 藤井正昭. 負荷分散クローラ機構の開発. 
	日本ロボット学会誌. 1987, vol.5, pp.335-388. 
	\url{https://doi.org/10.7210/jrsj.5.335}
	(閲覧日2023-1-10)

	\bibitem{magnetic-clawlar3}
	日立製作所 歴史1990年代のロボット 1992年 磁気クローラ式検査ロボット
	\url{https://www.hitachi.co.jp/rd/research/mechanical/robotics/history/1990.html}
	(閲覧日2023-1-10)

	\bibitem{meca-s}
	綱川大悟. 全方位移動可能な壁面走行ロボットの開発. 千葉工業大学. 修士学位論文.

	\begin{comment}
	
	\bibitem{dia}
	ダイヤ電子応用株式会社 磁気クローラロボット
	\url{https://www.dia-elec.com/format/jikikuroura.html}
	(閲覧日2023-1-14)
	\end{comment}

	\bibitem{MC}
	三菱ケミカルアドバンスドマテリアルズ株式会社 MCナイロン 技術資料
	\url{https://media.mcam.com/fileadmin/quadrant/documents/QEPP/AP/Japan/Brochures/MC_technical_data.pdf}
	(閲覧日2023-1-16)

	\bibitem{moon}
	別府 伸耕 著,月着陸船アポロに学ぶ確率統計コンピュータ,トランジスタ技術,2019年7月号 特集,CQ出版社.

	\bibitem{magfine67}
	株式会社マグファイン NSC0058 ネオジム 67mm X 10mm X 5mm /M3
	\url{https://www.magfine.com/products/detail/7288}
	(閲覧日2023-1-10)

	\bibitem{magfine60} 
	株式会社マグファイン NSC0075 ネオジム 60mm X 10mm X 5mm /M3
	\url{https://www.magfine.com/products/detail/8206}
	(閲覧日2023-1-10)

	\bibitem{magfine_faq}
	株式会社マグファイン よくある質問 耐熱温度とは
	\url{https://www.magfine.com/support/specification/518/}
	(閲覧日2023-1-10)

	\bibitem{magfine_cal}
	株式会社マグファイン 磁気計算機
	\url{https://www.magfine.com/tool/design}
	(閲覧日2023-1-10)

	\bibitem{pura}
	本間精一, プラスチック材料大全, 日刊工業新聞社.

	\bibitem{takano}
	高野菊雄, これでわかるプラスチック技術, 技術評論社.

	\bibitem{nezi}
	株式会社ミスミ 技術資料 台形ねじ 技術計算
	\url{https://jp.misumi-ec.com/tech-info/categories/technical_data/td03/a0071.html}
	(閲覧日2023-1-16)

	\bibitem{zyusi}
	株式会社ミスミ 樹脂・セラミックス 特性値一覧
	\url{https://jp.misumi-ec.com/maker/misumi/mech_material/products/properties/resin/}
	(閲覧日2023-1-16)

\end{thebibliography}



%謝辞
%謝辞
\lhead {}   %:天左側ヘッダの定義
\newpage
\rhead {謝辞}   %:天右側ヘッダの定義
\section* {謝辞}
\addcontentsline {toc} {section} {謝辞}
\vspace {5mm}

本研究を進めるにあたりご指導くださいました米田完教授に謹んで感謝の意を表します. 先生の多くのご助言があったからこそ, ここまで研究を進めることができました. 


共同研究先である日本橋梁株式会社の中原様, 高木様にも大変お世話になりました. 心より感謝申し上げます.


また, 米田研究室の皆様にも日ごろから実験や製作の手伝いをして頂きました. また, 助言もいただきました. 心より感謝します.


最後に, 修士課程に進む機会を与えてくださり, あらゆる面で支えてくださった両親にこの場を借りて感謝申し上げます.
\clearpage

\includepdf[pages={-},nup=1x2]{pdf/A_body_plan.pdf}
\includepdf[pages={-},nup=1x2]{pdf/B_track-shoe_plan.pdf}
\includepdf[pages={-},nup=1x2]{pdf/C_sprocket-assy_plan.pdf}
\includepdf[pages={-},nup=1x2]{pdf/D_ono.pdf}
\includepdf[pages={-}]{pdf/E_quote.pdf}
\includepdf[pages={-},nup=1x2]{pdf/F_test_of_const_pub.pdf}

%\includepdf[pages={-}]{pdf/F_test_of_const.pdf}

%\includepdf[pages={-}]{pdf/G_NSC0058_SPEC.pdf}
%\includepdf[pages={-}]{pdf/H_NSC0075_SPEC.pdf}

\end{document}