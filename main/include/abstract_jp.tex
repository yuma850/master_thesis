 %←スペース用全角スペース()
\vspace{20mm}
\renewcommand{\abstractname}{\LARGE 要旨}
\begin{abstract}
  \vspace{3mm}
  本研究では鋼橋のメンテナンスにおける塗膜剥離作業のIH装置による加熱工程を担う磁気クローラ式壁面移動ロボットの開発を行う.
  鋼製の橋は防食性能を維持するために定期的な塗装の塗り替えが必要である. その際, 新たに塗装を行う前に劣化した既存の塗料を除去することが重要である. 現在, 塗膜剥離の方法の一つにIH(電磁誘導加熱)によるものがある. しかし, IHの装置は約12kgと重く, それを保持し操作する作業員の負担の軽減が求められている. そこで, 鋼橋の壁面を磁気クローラで吸着し, IHの装置を自動で移動させることができる壁面移動ロボットの開発した. 磁気クローラは, 履板に設けたピンをレールに沿わせることにより壁面に垂直方向の剛性を持たせる方式を採用している. これは先行研究を参考にした同様の仕組みだが, アキシアル荷重を受けることができる摺動面を設けることにより先行研究では論じられていない旋回動作を考慮した構造としている. 1/2スケールの試作機を製作し, 動作実験・検証を行い, IMUを用いた自律移動の実装も行った. これによって得られた知見をもとにフルスケールサイズの機体の設計製作行った. そして, 実際の橋梁に使用されているものと同様の試験桁において塗膜剥離試験を行い, その有用性を示すとともに課題を明らかにした.
\end{abstract}

\vspace{20mm}
\renewcommand{\abstractname}{\LARGE Abstract}
\begin {abstract}
  In this study, a magnetic crawler-type wall locomotion robot is developed to perform the heating process by IH equipment in the coating removal process in the maintenance of steel bridges.
  Steel bridges require regular repainting to prevent corrosion. 
  When repainting, it is important to remove the deteriorated existing paint before applying new paint. 
  Currently, one method of coating removal is through IH (induction heating).
  However, IH devices are heavy, approximately 12kg, and there is a need to reduce the burden on the workers who operate and hold them.
  Therefore, a wall locomotion robot that can automatically move the IH device by adhering the steel bridge wall surface with a magnetic crawler has been developed.
  The magnetic crawlers are designed to provide vertical rigidity to the wall surface by aligning the pins on the track with the rails. 
  This is a similar mechanism based on previous studies, but by providing a sliding surface that can withstand axial loads, it is a structure that considers turning movement not discussed in previous studies. 
  A half scale prototype was produced and operation experiments and verification were conducted, and autonomous movement using IMU was also implemented.
  Based on the knowledge obtained, the design and production of a full scale robo was carried out.
  Then, The results of the peeling tests on the test girders similar to those used in actual steel bridges show the usefulness of the method and clarify the issues involved.

  \end {abstract}

