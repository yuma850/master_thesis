%第3章
\newpage
\part{結論}

\section{まとめ}
本研究では, 鋼橋における塗装塗り替え時の塗膜剥離作業において, IH塗膜剥離装置を壁面を自動走行させるロボットの開発を行った. 現場と近い環境における塗膜剥離試験にて塗膜剥離に必要な加熱を行うことができることを確認した. 

第\ref{zyoron}章\ref{goal}の設計要件の一つである半自動加熱については,IHヘッド電源投入時に誤動作を起こしたため加熱を伴う試験を行うことができなかった.自律移動は行うことができたため,回路のノイズ対策を行うことで解決すると考えた. 
その他の研究目標,設計要件は達成することができた.
さらに, いくつかの課題が明らかになった.

\section{課題}
\subsection{履板や磁石の熱による影響}
\ref{netu}で明らかになったように, 現状の機体は磁石や履板の材料が熱の影響を受ける可能性がある.
よって, より耐熱性のある磁石や材料を検討する必要がある.

\subsection{補強材や桁間における取り付け・取り外し}
鋼橋の桁には多く補剛材が設置されており, 本研究ではその乗り越えを目標としていない. 
以下に塗膜剥離試験で使用した桁の裏側(補剛材のある面)の写真を示す.
運用にあたっては1橋梁の施工にてロボットを何度も取り付け及び取り外しを行う必要がある. 
ロボットの重量が重いほど, 作業員の負担となるため軽量化が求められる. 
また, 容易に補強材を超えたロボットの移動ができるような工夫が求められる. 

\begin{figure}[H]
  \centering
  \includegraphics[width=0.6\textwidth]{const/ura.JPG}
  \caption{Side of bridge girder with reinforcement}
  \label{keta-ura}
\end{figure}

\section{今後の展望}
実用化, 及び実際の橋梁での塗膜り試験に向けて, 株式会社日南が新機体の開発を行うこととなった. 
開発にあったっては先述の課題も踏まえ, 要改良点として次の事項を念頭に置くこととした.

\begin{itemize}
  \item 機体の小型軽量化
  \item 履板及び磁石の耐熱化
  \item IHによる回路誤作動の対策
  \item IHヘッドの位置の調整の容易化
\end{itemize}

%小形軽量化にあたっては, 履板のより摩擦係数の高い材質を用い, 必要な吸着力を減らすことで磁石自体の小形軽量化を図ることができると考える.

履板の材質については, ピンの固定部と壁面接触部の材質を分けるなどの工夫をすることで, 今回は加工の観点から使用しなかったポリカーボネートなどの材料を使うことができると考えられるる.

さらにフレーム磁石についてはヨークのほかにハルバッハ配列など磁石の配列についても工夫を行い吸着力を向上させることができると考えられる. またこのような工夫を容易に行えるフレーム磁石のみに磁石を配置し, 履板には磁石を使用せず小型軽量化するといった案も考えられる. 

\section{発展的な展望}
今回研究した磁気クローラを応用し, 面間遷移動作を目指した機体を検討した(図\ref{future}).
正方形の磁気クローラユニットを4ユニット用い, 受動関節で接続することで床, 壁面及び天井間の移動ができと考えられる.

\begin{figure}[H]
  \centering
  \includegraphics[width=0.8\textwidth]{future.png}
  \caption{Application of magnetic crawlers$\colon$ Robot that can move between walls and ceilings}
  \label{future}
\end{figure}






